\documentclass[a4paper,10pt]{article}
\usepackage[latin1]{inputenc}
\usepackage[brazil]{babel}
\usepackage{babel}
\usepackage{graphics}
\usepackage{color}
\usepackage{epsfig}
\usepackage{alltt,fancyvrb}
\usepackage{listings}
\usepackage{float,ctable}
\usepackage{dentitle}
\usepackage{personal}

\renewcommand{\rmdefault}{cmss}

\begin{document}

\title{Exerc�cios sobre Clustering com R - EP3}
\author{Prof. Fabr�cio Jailson Barth}
\date{$1^{o}$ semestre de 2015}
\makedendentitle{Faculdade de Tecnologia Bandeirantes -
  BandTec}{Sistemas Inteligentes}{}

\section{Quest�es sobre clustering}

\begin{enumerate}
\item Utilizando o dataset \textbf{ruspini} do pacote
  \textbf{cluster}, execute uma an�lise de cluster utilizando o
  algoritmo \textbf{k-means}.
\item Fa�a uma an�lise de cluster utilizando o dataset sobre
  \textbf{abalos s�smicos} do site
  \footnote{http://earthquake.usgs.gov/earthquakes/feed/v1.0/summary/all\_month.csv}. Considere
  apenas as vari�veis de \textbf{profundidade} e \textbf{magnitude}. 
\item Imprima a informa��o dos clusters em um mapa
  georeferenciado. Para imprimir pontos em um mapa georeferenciado
  voc� pode utilizar o c�digo abaixo:

\begin{alltt}
library(maps)
library(mapdata)
map(mar = c(0.1, 0.1, 0.1, 0.1), myborder=0.00001)
points(abalos$longitude, abalos$latitude, col=2, pch=20)
\end{alltt}

Voc� identificou algum padr�o que voc� considere relevante?

\item Fa�a uma an�lise de cluster utilizando o dataset
  \textbf{survey}, atributos \textbf{Age} e \textbf{Height},
  do pacote \textbf{UsingR}. Implemente tr�s cen�rios diferentes: 
\begin{enumerate}
\item Com os valores originais;
\item Com os valores da altura (Height) em metros, e;
\item Com os valores da altura (Height) e idade (Age) devidamente
  normalizados. 
\end{enumerate} 

Comente o que acontece em cada um dos casos.

\item Utilizando o dataset \textbf{survey}, atributos \textbf{Exer}
  (sobre o h�bito de fazer exerc�cios) e \textbf{Smoke} (sobre o
  h�bito de fumar), fa�a uma an�lise de cluster deste dataset. 

\item Levando-se em considera��o o dataset do item anterior, inclua o
  atributo de sexo (\textbf{Sex}) e fa�a a mesma an�lise.

\end{enumerate}

\section{C�digos que podem ser �teis}

\begin{alltt}
elbow <- function(dataset)\{
  wss <- numeric(15)
  for (i in 1:15) 
    wss[i] <- sum(kmeans(dataset,centers=i, 
                  nstart=100)$withinss)
  plot(1:15, wss, type="b", main="Elbow method", 
       xlab="Number of Clusters",
       ylab="Within groups sum of squares", 
       pch=8, col="red")
\}
\end{alltt}

\section{M�todo de entrega}

Este trabalho dever� ser realizado em dupla. A data m�xima para
entrega � no dia 26 de maio de 2015. A entrega deve ser realizada em
sala de aula.

\end{document}










