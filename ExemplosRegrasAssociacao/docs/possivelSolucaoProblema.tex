\documentclass[a4paper,10pt]{article}
\usepackage[latin1]{inputenc}
\usepackage[brazil]{babel}
\usepackage{babel}
\usepackage{graphics}
\usepackage{color}
\usepackage{epsfig}
\usepackage{alltt,fancyvrb}
\usepackage{listings}
\usepackage{float,ctable}
\usepackage{dentitle}
\usepackage{personal}

\graphicspath{ {../results/} }

\renewcommand{\rmdefault}{cmss}

\usepackage{listings}
\usepackage{color}

\definecolor{dkgreen}{rgb}{0,0.6,0}
\definecolor{gray}{rgb}{0.5,0.5,0.5}
\definecolor{mauve}{rgb}{0.58,0,0.82}

\lstset{frame=tb,
  language=R,
  aboveskip=5mm,
  belowskip=5mm,
  showstringspaces=false,
  columns=flexible,
  basicstyle={\small\ttfamily},
  numbers=left,
  numberstyle=\tiny\color{gray},
  keywordstyle=\color{blue},
  commentstyle=\color{dkgreen},
  stringstyle=\color{mauve},
  breaklines=true,
  breakatwhitespace=true,
  tabsize=3
}

\begin{document}

\title{Descobrindo regras de associa��o entre tipos de crimes
  realizados na cidade de Los Angeles - 
  poss�vel solu��o}
\author{Prof. Fabr�cio Jailson Barth}
\date{$1^{o}$ semestre de 2015}
\makedendentitle{Faculdade de Tecnologia Bandeirantes - BandTec}{P�s Gradua��o
  em Big Data}{}

\section{Introdu��o}

O objetivo deste trabalho � identificar regras de associa��o entre
tipos de crimes que aconteceram na cidade de Los Angeles entre $1^{o}$
de novembro e 31 de dezembro de 2014. Neste trabalho as ocorr�ncias de
crimes est�o agrupadas levando-se em considera��o a regi�o onde o
crime aconteceu, a data e o hor�rio. O dataset original pode ser
encontrado nesta URL\footnote{https://raw.githubusercontent.com/fbarth/posGraduacao/master/ExemplosRegrasAssociacao/dataset/LAPD Crime and Collision Raw Data -
2014.csv}. No entanto, este inicia a partir do dataset j�
transformado\footnote{https://raw.githubusercontent.com/fbarth/posGraduacao/master/ExemplosRegrasAssociacao/dataset/temp_trans.csv},
transforma��o esta que est� documentada no enunciado do projeto.

\section{M�todo}

O m�todo utilizado por esta an�lise � composto pelas seguintes fases:
aquisi��o e pr�-processamento dos dados; constru��o do modelo, e;
an�lise do modelo. 

\subsection{Aquisi��o e pr�-processamento dos dados}



\subsection{Constru��o do modelo}


\subsection{An�lise do modelo}


\section{Considera��es}



\end{document}










